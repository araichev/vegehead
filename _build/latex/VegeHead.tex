% Generated by Sphinx.
\def\sphinxdocclass{report}
\documentclass[letterpaper,10pt,english]{sphinxmanual}
\usepackage[utf8]{inputenc}
\DeclareUnicodeCharacter{00A0}{\nobreakspace}
\usepackage{cmap}
\usepackage[T1]{fontenc}
\usepackage{babel}
\usepackage{times}
\usepackage[Bjarne]{fncychap}
\usepackage{longtable}
\usepackage{sphinx}
\usepackage{multirow}


\title{VegeHead Documentation}
\date{January 01, 2014}
\release{2.0}
\author{Alex Raichev}
\newcommand{\sphinxlogo}{}
\renewcommand{\releasename}{Release}
\makeindex

\makeatletter
\def\PYG@reset{\let\PYG@it=\relax \let\PYG@bf=\relax%
    \let\PYG@ul=\relax \let\PYG@tc=\relax%
    \let\PYG@bc=\relax \let\PYG@ff=\relax}
\def\PYG@tok#1{\csname PYG@tok@#1\endcsname}
\def\PYG@toks#1+{\ifx\relax#1\empty\else%
    \PYG@tok{#1}\expandafter\PYG@toks\fi}
\def\PYG@do#1{\PYG@bc{\PYG@tc{\PYG@ul{%
    \PYG@it{\PYG@bf{\PYG@ff{#1}}}}}}}
\def\PYG#1#2{\PYG@reset\PYG@toks#1+\relax+\PYG@do{#2}}

\expandafter\def\csname PYG@tok@gd\endcsname{\def\PYG@tc##1{\textcolor[rgb]{0.63,0.00,0.00}{##1}}}
\expandafter\def\csname PYG@tok@gu\endcsname{\let\PYG@bf=\textbf\def\PYG@tc##1{\textcolor[rgb]{0.50,0.00,0.50}{##1}}}
\expandafter\def\csname PYG@tok@gt\endcsname{\def\PYG@tc##1{\textcolor[rgb]{0.00,0.27,0.87}{##1}}}
\expandafter\def\csname PYG@tok@gs\endcsname{\let\PYG@bf=\textbf}
\expandafter\def\csname PYG@tok@gr\endcsname{\def\PYG@tc##1{\textcolor[rgb]{1.00,0.00,0.00}{##1}}}
\expandafter\def\csname PYG@tok@cm\endcsname{\let\PYG@it=\textit\def\PYG@tc##1{\textcolor[rgb]{0.25,0.50,0.56}{##1}}}
\expandafter\def\csname PYG@tok@vg\endcsname{\def\PYG@tc##1{\textcolor[rgb]{0.73,0.38,0.84}{##1}}}
\expandafter\def\csname PYG@tok@m\endcsname{\def\PYG@tc##1{\textcolor[rgb]{0.13,0.50,0.31}{##1}}}
\expandafter\def\csname PYG@tok@mh\endcsname{\def\PYG@tc##1{\textcolor[rgb]{0.13,0.50,0.31}{##1}}}
\expandafter\def\csname PYG@tok@cs\endcsname{\def\PYG@tc##1{\textcolor[rgb]{0.25,0.50,0.56}{##1}}\def\PYG@bc##1{\setlength{\fboxsep}{0pt}\colorbox[rgb]{1.00,0.94,0.94}{\strut ##1}}}
\expandafter\def\csname PYG@tok@ge\endcsname{\let\PYG@it=\textit}
\expandafter\def\csname PYG@tok@vc\endcsname{\def\PYG@tc##1{\textcolor[rgb]{0.73,0.38,0.84}{##1}}}
\expandafter\def\csname PYG@tok@il\endcsname{\def\PYG@tc##1{\textcolor[rgb]{0.13,0.50,0.31}{##1}}}
\expandafter\def\csname PYG@tok@go\endcsname{\def\PYG@tc##1{\textcolor[rgb]{0.20,0.20,0.20}{##1}}}
\expandafter\def\csname PYG@tok@cp\endcsname{\def\PYG@tc##1{\textcolor[rgb]{0.00,0.44,0.13}{##1}}}
\expandafter\def\csname PYG@tok@gi\endcsname{\def\PYG@tc##1{\textcolor[rgb]{0.00,0.63,0.00}{##1}}}
\expandafter\def\csname PYG@tok@gh\endcsname{\let\PYG@bf=\textbf\def\PYG@tc##1{\textcolor[rgb]{0.00,0.00,0.50}{##1}}}
\expandafter\def\csname PYG@tok@ni\endcsname{\let\PYG@bf=\textbf\def\PYG@tc##1{\textcolor[rgb]{0.84,0.33,0.22}{##1}}}
\expandafter\def\csname PYG@tok@nl\endcsname{\let\PYG@bf=\textbf\def\PYG@tc##1{\textcolor[rgb]{0.00,0.13,0.44}{##1}}}
\expandafter\def\csname PYG@tok@nn\endcsname{\let\PYG@bf=\textbf\def\PYG@tc##1{\textcolor[rgb]{0.05,0.52,0.71}{##1}}}
\expandafter\def\csname PYG@tok@no\endcsname{\def\PYG@tc##1{\textcolor[rgb]{0.38,0.68,0.84}{##1}}}
\expandafter\def\csname PYG@tok@na\endcsname{\def\PYG@tc##1{\textcolor[rgb]{0.25,0.44,0.63}{##1}}}
\expandafter\def\csname PYG@tok@nb\endcsname{\def\PYG@tc##1{\textcolor[rgb]{0.00,0.44,0.13}{##1}}}
\expandafter\def\csname PYG@tok@nc\endcsname{\let\PYG@bf=\textbf\def\PYG@tc##1{\textcolor[rgb]{0.05,0.52,0.71}{##1}}}
\expandafter\def\csname PYG@tok@nd\endcsname{\let\PYG@bf=\textbf\def\PYG@tc##1{\textcolor[rgb]{0.33,0.33,0.33}{##1}}}
\expandafter\def\csname PYG@tok@ne\endcsname{\def\PYG@tc##1{\textcolor[rgb]{0.00,0.44,0.13}{##1}}}
\expandafter\def\csname PYG@tok@nf\endcsname{\def\PYG@tc##1{\textcolor[rgb]{0.02,0.16,0.49}{##1}}}
\expandafter\def\csname PYG@tok@si\endcsname{\let\PYG@it=\textit\def\PYG@tc##1{\textcolor[rgb]{0.44,0.63,0.82}{##1}}}
\expandafter\def\csname PYG@tok@s2\endcsname{\def\PYG@tc##1{\textcolor[rgb]{0.25,0.44,0.63}{##1}}}
\expandafter\def\csname PYG@tok@vi\endcsname{\def\PYG@tc##1{\textcolor[rgb]{0.73,0.38,0.84}{##1}}}
\expandafter\def\csname PYG@tok@nt\endcsname{\let\PYG@bf=\textbf\def\PYG@tc##1{\textcolor[rgb]{0.02,0.16,0.45}{##1}}}
\expandafter\def\csname PYG@tok@nv\endcsname{\def\PYG@tc##1{\textcolor[rgb]{0.73,0.38,0.84}{##1}}}
\expandafter\def\csname PYG@tok@s1\endcsname{\def\PYG@tc##1{\textcolor[rgb]{0.25,0.44,0.63}{##1}}}
\expandafter\def\csname PYG@tok@gp\endcsname{\let\PYG@bf=\textbf\def\PYG@tc##1{\textcolor[rgb]{0.78,0.36,0.04}{##1}}}
\expandafter\def\csname PYG@tok@sh\endcsname{\def\PYG@tc##1{\textcolor[rgb]{0.25,0.44,0.63}{##1}}}
\expandafter\def\csname PYG@tok@ow\endcsname{\let\PYG@bf=\textbf\def\PYG@tc##1{\textcolor[rgb]{0.00,0.44,0.13}{##1}}}
\expandafter\def\csname PYG@tok@sx\endcsname{\def\PYG@tc##1{\textcolor[rgb]{0.78,0.36,0.04}{##1}}}
\expandafter\def\csname PYG@tok@bp\endcsname{\def\PYG@tc##1{\textcolor[rgb]{0.00,0.44,0.13}{##1}}}
\expandafter\def\csname PYG@tok@c1\endcsname{\let\PYG@it=\textit\def\PYG@tc##1{\textcolor[rgb]{0.25,0.50,0.56}{##1}}}
\expandafter\def\csname PYG@tok@kc\endcsname{\let\PYG@bf=\textbf\def\PYG@tc##1{\textcolor[rgb]{0.00,0.44,0.13}{##1}}}
\expandafter\def\csname PYG@tok@c\endcsname{\let\PYG@it=\textit\def\PYG@tc##1{\textcolor[rgb]{0.25,0.50,0.56}{##1}}}
\expandafter\def\csname PYG@tok@mf\endcsname{\def\PYG@tc##1{\textcolor[rgb]{0.13,0.50,0.31}{##1}}}
\expandafter\def\csname PYG@tok@err\endcsname{\def\PYG@bc##1{\setlength{\fboxsep}{0pt}\fcolorbox[rgb]{1.00,0.00,0.00}{1,1,1}{\strut ##1}}}
\expandafter\def\csname PYG@tok@kd\endcsname{\let\PYG@bf=\textbf\def\PYG@tc##1{\textcolor[rgb]{0.00,0.44,0.13}{##1}}}
\expandafter\def\csname PYG@tok@ss\endcsname{\def\PYG@tc##1{\textcolor[rgb]{0.32,0.47,0.09}{##1}}}
\expandafter\def\csname PYG@tok@sr\endcsname{\def\PYG@tc##1{\textcolor[rgb]{0.14,0.33,0.53}{##1}}}
\expandafter\def\csname PYG@tok@mo\endcsname{\def\PYG@tc##1{\textcolor[rgb]{0.13,0.50,0.31}{##1}}}
\expandafter\def\csname PYG@tok@mi\endcsname{\def\PYG@tc##1{\textcolor[rgb]{0.13,0.50,0.31}{##1}}}
\expandafter\def\csname PYG@tok@kn\endcsname{\let\PYG@bf=\textbf\def\PYG@tc##1{\textcolor[rgb]{0.00,0.44,0.13}{##1}}}
\expandafter\def\csname PYG@tok@o\endcsname{\def\PYG@tc##1{\textcolor[rgb]{0.40,0.40,0.40}{##1}}}
\expandafter\def\csname PYG@tok@kr\endcsname{\let\PYG@bf=\textbf\def\PYG@tc##1{\textcolor[rgb]{0.00,0.44,0.13}{##1}}}
\expandafter\def\csname PYG@tok@s\endcsname{\def\PYG@tc##1{\textcolor[rgb]{0.25,0.44,0.63}{##1}}}
\expandafter\def\csname PYG@tok@kp\endcsname{\def\PYG@tc##1{\textcolor[rgb]{0.00,0.44,0.13}{##1}}}
\expandafter\def\csname PYG@tok@w\endcsname{\def\PYG@tc##1{\textcolor[rgb]{0.73,0.73,0.73}{##1}}}
\expandafter\def\csname PYG@tok@kt\endcsname{\def\PYG@tc##1{\textcolor[rgb]{0.56,0.13,0.00}{##1}}}
\expandafter\def\csname PYG@tok@sc\endcsname{\def\PYG@tc##1{\textcolor[rgb]{0.25,0.44,0.63}{##1}}}
\expandafter\def\csname PYG@tok@sb\endcsname{\def\PYG@tc##1{\textcolor[rgb]{0.25,0.44,0.63}{##1}}}
\expandafter\def\csname PYG@tok@k\endcsname{\let\PYG@bf=\textbf\def\PYG@tc##1{\textcolor[rgb]{0.00,0.44,0.13}{##1}}}
\expandafter\def\csname PYG@tok@se\endcsname{\let\PYG@bf=\textbf\def\PYG@tc##1{\textcolor[rgb]{0.25,0.44,0.63}{##1}}}
\expandafter\def\csname PYG@tok@sd\endcsname{\let\PYG@it=\textit\def\PYG@tc##1{\textcolor[rgb]{0.25,0.44,0.63}{##1}}}

\def\PYGZbs{\char`\\}
\def\PYGZus{\char`\_}
\def\PYGZob{\char`\{}
\def\PYGZcb{\char`\}}
\def\PYGZca{\char`\^}
\def\PYGZam{\char`\&}
\def\PYGZlt{\char`\<}
\def\PYGZgt{\char`\>}
\def\PYGZsh{\char`\#}
\def\PYGZpc{\char`\%}
\def\PYGZdl{\char`\$}
\def\PYGZhy{\char`\-}
\def\PYGZsq{\char`\'}
\def\PYGZdq{\char`\"}
\def\PYGZti{\char`\~}
% for compatibility with earlier versions
\def\PYGZat{@}
\def\PYGZlb{[}
\def\PYGZrb{]}
\makeatother

\begin{document}

\maketitle
\tableofcontents
\phantomsection\label{index::doc}


This is a cookbook in progress. In compiling it, i set myself the following requirements.

\textbf{Yummy.} The food should taste great.

\textbf{Fast.} The food should take at most 30 minutes to make. I enjoy cooking but don’t want to spend a lot of time doing it. To fulfill this requirement, several of the recipes herein use a pressure cooker.

\textbf{Vegetarian.} The food should not require the killing of animals. I like to minimize the animal suffering involved in my meals and going vegetarian is a step in the right direction. Going vegan is a bigger step.

\textbf{Pre-industrial.} The ingredients should be as close as possible to the ones eaten by our pre-industrial ancestors. That rules out 20th-century edible food-like substances (synthetic flavorings, high-fructose corn syrup, skim milk powder, etc.) and food grown with synthetic pesticides. I don’t trust that stuff. I’ll let canned foods slide, though.

\textbf{Grain-free.} The food should not contain grains: wheat, rice, corn, oats, etc. I’m lowering my carbohydrate intake, because i’ve read that it’s healthy, and cutting out grains is one simple way to do that. I’m cutting back on sugars too but still include some sweet recipes. Ice cream, mmm.

In formatting this book, i abbreviated the standard cooking measurements in the following way:

C = cup, T = tablespoon, t = teaspoon.

I also used metric units such as liters, kilograms, and Celsius with their standard abbreviations. Finally, i chose to write single fractions instead of mixed numbers, such as 3/2 instead of 1 1/2, because i think they look better.

This book was created using \href{http://sphinx.pocoo.org/}{Sphinx} and is licensed under the Creative Commons Attribution-ShareAlike license. Enjoy and share!

\includegraphics{by-sa.png}


\chapter{Tempeh Sauerkraut}
\label{tempeh_sauerkraut:introduction}\label{tempeh_sauerkraut:tempeh-sauerkraut}\label{tempeh_sauerkraut::doc}

\section{Ingredients}
\label{tempeh_sauerkraut:ingredients}\begin{itemize}
\item {} 
1   T   coconut oil

\item {} 
1       large onion, julienned

\item {} 
1       red capsicum, diced

\item {} 
2       cloves garlic, diced

\item {} 
1/2 t   caraway seeds

\item {} 
500 g   tempeh, cut into fingers

\item {} 
3   C   sauerkraut

\item {} 
1/4 C   water

\item {} 
3   T   homemade mayonnaise

\end{itemize}


\section{Directions}
\label{tempeh_sauerkraut:directions}\begin{enumerate}
\item {} 
Heat the oil and sauté the tempeh, capsicum, and onion until the tempeh is browned on one side, about 4 minutes.

\item {} 
Flip the tempeh fingers, add the rest of the ingredients except the mayonnaise, and continue to cook over medium-high heat, stirring, until the tempeh is browned on the other side.

\item {} 
Add the mayonnaise and serve.

\end{enumerate}


\section{Notes}
\label{tempeh_sauerkraut:notes}
Serves 3.


\chapter{Mung Dal}
\label{mung_dal:mung-dal}\label{mung_dal::doc}

\section{Ingredients}
\label{mung_dal:ingredients}\begin{itemize}
\item {} 
3/2 C     mung beans

\item {} 
5 C       water

\item {} 
2 T       coconut oil

\item {} 
400mL     chopped tomatoes

\item {} 
1         onion, chopped

\item {} 
2         cloves garlic, chopped

\item {} 
1 T       shredded ginger

\item {} 
1 t       ground turmeric

\item {} 
1         hot chili

\item {} 
1         red capsicum, chopped

\item {} 
1 t       salt

\item {} 
1 t       cumin seeds

\item {} 
2 t       black mustard seeds

\item {} 
some      coriander leaf for garnish

\end{itemize}


\section{Directions}
\label{mung_dal:directions}\begin{enumerate}
\item {} 
Except for 1 T coconut oil, the cumin seeds, and the mustard seeds, put everything into a pressure cooker, and cook at high pressure for 11 minutes.

\item {} 
Fry the reserved coconut oil, cumin seeds, and mustard seeds until the mustard seeds begin to pop.

\item {} 
Add the fried seeds and oil to the soup, garnish with coriander leaf, and serve.

\end{enumerate}


\section{Notes}
\label{mung_dal:notes}
Serves about 6.


\chapter{Chickpea Curry}
\label{chickpea_curry::doc}\label{chickpea_curry:chickpea-curry}

\section{Ingredients}
\label{chickpea_curry:ingredients}\begin{itemize}
\item {} 
1   C   dry chickpeas, soaked for at least 4 hours

\item {} 
3   C   water

\item {} 
1   T   coconut oil

\item {} 
1/2     t   ground cinnamon

\item {} 
1   T   fennel seeds

\item {} 
1/8     C   curry leaves

\item {} 
1/2     T   asafetida

\item {} 
400     mL  coconut cream

\item {} 
400     mL  chopped tomato

\item {} 
1/2         kumara, diced

\item {} 
1/2     T   ground fennel seeds

\item {} 
3/4     T   ground coriander seeds

\item {} 
3/4     T   ground fenugreek

\item {} 
1/2     T   ground turmeric

\item {} 
1/2     t   chili powder

\item {} 
1/4     T   garam masala

\item {} 
1   T   channa masala

\item {} 
1       head broccoli, chopped

\item {} 
1   t   tamarind concentrate

\item {} 
1   t   salt or to taste

\item {} 
some        chopped coriander for garnish

\end{itemize}


\section{Directions}
\label{chickpea_curry:directions}\begin{enumerate}
\item {} 
Pressure cook the chickpeas and water at high pressure for 18 minutes, separate the liquid from the beans, and set aside both.

\item {} 
In the meantime, in a large saucepan, heat the coconut oil and sauté the cinnamon, unground fennel seeds, curry leaves, and asafetida until fragrant.

\item {} 
Add the coconut cream, chopped tomato, and kumara and bring to a boil.

\item {} 
Reduce heat to simmer and add the remaining spices and kumara.

\item {} 
Cook until the kumara is just short of tender, about 8 minutes.

\item {} 
Add the broccoli and tamarind and cook for another 5 minutes.

\item {} 
Add the cooked chickpeas and enough chickpea liquid to reach your desired consistency.

\item {} 
Garnish with coriander and serve.

\end{enumerate}


\section{Notes}
\label{chickpea_curry:notes}
Serves 4. This is the spiciest (highest spice count) dish i’ve ever cooked! It is also my favorite Hare Krishna recipe.


\chapter{Chili}
\label{chili:chili}\label{chili::doc}

\section{Ingredients}
\label{chili:ingredients}\begin{itemize}
\item {} 
3   T   olive oil

\item {} 
1       onion, diced

\item {} 
3       cloves garlic, diced

\item {} 
1       green capsicum, diced

\item {} 
2       jalapeños, diced

\item {} 
2       ribs celery, diced

\item {} 
1   t   oregano

\item {} 
1   t   basil

\item {} 
1   t   cumin seed

\item {} 
1   t   smoked paprika

\item {} 
1/2 t   ground turmeric

\item {} 
1/4 t   cinnamon

\item {} 
5   C   water

\item {} 
400 mL  chopped tomatoes

\item {} 
2   C   assorted dry beans (but not chickpeas, because they take

\item {} 
too long to cook compared to other beans),

\item {} 
soaked for at least 4 hours

\item {} 
1/2     kumara, diced

\item {} 
2   C   greens such as kale

\item {} 
1   t   salt or to taste

\item {} 
some        coriander for garnish

\end{itemize}


\section{Directions}
\label{chili:directions}\begin{enumerate}
\item {} 
In a pressure cooker, heat 1 T olive oil and sauté the onion, garlic, capsicum, and chilis until golden.

\item {} 
Add in everything else except the salt and coriander, and pressure cook at high pressure for however long the slowest cooking bean needs, e.g. 12 minutes for a kidney bean + pinto bean + mung bean combination.

\item {} 
Add the remaining olive oil, salt, and coriander, and serve.

\end{enumerate}


\section{Notes}
\label{chili:notes}
Serves 6.


\chapter{Italian Lentil Soup}
\label{italian_lentil_soup:italian-lentil-soup}\label{italian_lentil_soup::doc}

\section{Ingredients}
\label{italian_lentil_soup:ingredients}\begin{itemize}
\item {} 
6       C       water

\item {} 
2   C   brown lentils, rinsed

\item {} 
2   T   olive oil

\item {} 
1       large onion, peeled and coarsely chopped

\item {} 
2       large cloves garlic, peeled and minced

\item {} 
4       ribs celery, cut into 2cm slices

\item {} 
2       large carrots, chopped

\item {} 
125 g   mushrooms, sliced

\item {} 
2       bay leaves

\item {} 
1   t   dried thyme or marjoram

\item {} 
3/4 t   dried oregano

\item {} 
1/4 t   dried chili flakes

\item {} 
3   T   tomato paste or 2 large tomatoes, coarsely chopped

\item {} 
2   T   balsamic vinegar

\item {} 
1   t   salt or to taste

\item {} 
some        parsley for garnish

\end{itemize}


\section{Directions}
\label{italian_lentil_soup:directions}\begin{enumerate}
\item {} 
Put everything except the tomato paste, vinegar, and salt in a pressure cooker, and pressure cook everything at high pressure for 11 minutes.

\item {} 
Remove the bay leaves, add the vinegar and salt, dissolve the tomato paste in a cup of soup, and stir the cup back into the soup. Garnish with parsley and serve.

\end{enumerate}


\section{Notes}
\label{italian_lentil_soup:notes}
Serves 6.


\chapter{Orange Squash Soup}
\label{orange_squash_soup:orange-squash-soup}\label{orange_squash_soup::doc}

\section{Ingredients}
\label{orange_squash_soup:ingredients}\begin{itemize}
\item {} 
1   kg  butternut squash, kabocha, or delicate squash, scrubbed,        - seeded, and cut into 1 cm chunks (peeling not necessary)

\item {} 
1       small onion, peeled and coarsely chopped

\item {} 
2   C   cups water

\item {} 
2   T   coconut oil

\item {} 
1   C   freshly squeezed orange juice (see cook’s Notes)

\item {} 
1/4     C   rolled oats

\item {} 
1   T   freshly grated ginger

\item {} 
1   T   finely minced or grated orange peel

\item {} 
1/2     t   ground cinnamon

\item {} 
1/4     t   ground coriander seeds

\item {} 
1/2     t   sea salt, or to taste

\item {} 
1--2 T  maple syrup or honey

\item {} 
some        toasted pumpkin seeds for garnish

\end{itemize}


\section{Directions}
\label{orange_squash_soup:directions}\begin{enumerate}
\item {} 
Place all ingredients except the maple syrup into a pressure cooker, and cook at high pressure for 5 minutes.

\item {} 
Puree the soup with an imersion blender, add the maple syrup, thin slightly with water or orange juice if necessary, and garnish with toasted pumpkin seeds.

\end{enumerate}


\section{Notes}
\label{orange_squash_soup:notes}
Serves 4.


\chapter{Cabbage Salad}
\label{cabbage_salad:cabbage-salad}\label{cabbage_salad::doc}

\section{Ingredients}
\label{cabbage_salad:ingredients}\begin{itemize}
\item {} 
1       medium head cabbage (about 1.5 kg), thinly sliced

\item {} 
1       apple, grated

\item {} 
200 g   roasted and salted peanuts

\item {} 
1/2 C   desiccated coconut

\item {} 
1/4 C   lemon juice

\item {} 
1/4 C   chopped coriander

\item {} 
1/4 C   coconut oil

\item {} 
2   t   black mustard seed

\item {} 
2   t   cumin seed

\item {} 
2   t   asafetida

\item {} 
1/2 t   turmeric

\end{itemize}


\section{Directions}
\label{cabbage_salad:directions}\begin{enumerate}
\item {} 
Mix the cabbage, apple, peanuts, coconut, lemon juice, and coriander.

\item {} 
Heat the oil and fry the spices until fragrant.

\item {} 
Mix everything together and serve.

\end{enumerate}


\section{Notes}
\label{cabbage_salad:notes}
Serves 6.


\chapter{Red Cabbage Salad}
\label{red_cabbage_salad::doc}\label{red_cabbage_salad:red-cabbage-salad}

\section{Ingredients}
\label{red_cabbage_salad:ingredients}\begin{itemize}
\item {} 
1       medium head red cabbage (about 1.5 kg), grated

\item {} 
1   t   salt

\item {} 
1   t   caraway seeds

\item {} 
1       apple, grated

\item {} 
2/3 C   balsamic vinegar

\item {} 
2   T   olive oil

\end{itemize}


\section{Directions}
\label{red_cabbage_salad:directions}\begin{enumerate}
\item {} 
Combine the cabbage, salt, and caraway seeds in a large mixing bowl, and thoroughly squeeze the mixture for several minutes to soften the cabbage and release its juice.

\item {} 
Mix in the rest of the Ingredients.

\item {} 
If you have the time, chill the salad for several hours before serving.

\end{enumerate}


\section{Notes}
\label{red_cabbage_salad:notes}
Serves 6.


\chapter{Mashed Kumara}
\label{mashed_kumara::doc}\label{mashed_kumara:mashed-kumara}

\section{Ingredients}
\label{mashed_kumara:ingredients}\begin{itemize}
\item {} 
1    C      water

\item {} 
1       kg      kumara, cut into 1 cm cubes

\item {} 
2       T       coconut oil

\item {} 
1       onion, julienned

\item {} 
2   t       allspice

\item {} 
salt to taste

\end{itemize}


\section{Directions}
\label{mashed_kumara:directions}\begin{enumerate}
\item {} 
Use the water and a steamer basket to steam the kumara for about 10 minutes, or until soft.

\item {} 
Meanwhile, fry the onion in the coconut oil.

\item {} 
Mix the steamed kumara, fried onion, allspice, and salt, and mash.

\end{enumerate}


\chapter{Hummus}
\label{hummus:hummus}\label{hummus::doc}

\section{Ingredients}
\label{hummus:ingredients}\begin{itemize}
\item {} 
1   C   dry chickpeas, soaked for at least 4 hours

\item {} 
3   C   water

\item {} 
1/2  C  tahini

\item {} 
1/4  C  lemon juice

\item {} 
1/4  C  olive oil

\item {} 
1   t   salt

\item {} 
2       cloves garlic

\item {} 
1   t   cumin

\end{itemize}


\section{Directions}
\label{hummus:directions}
This is most easily prepared with a food processor, but you can also use simple implements such as a whisk, mortar, and pestle.
\begin{enumerate}
\item {} 
Pressure cook the chickpeas and water at high pressure for 18 minutes.

\item {} 
Separate the chickpeas and liquid and save both.

\item {} 
Whip the tahini and lemon juice.

\item {} 
Add in the remaining ingredients, blend, and thin to your desired consistency with the chickpea liquid.

\end{enumerate}


\section{Notes}
\label{hummus:notes}
Makes about 3 cups.


\chapter{Tarator}
\label{tarator:tarator}\label{tarator::doc}

\section{Ingredients}
\label{tarator:ingredients}\begin{itemize}
\item {} 
1   L       yoghurt

\item {} 
3   C       water

\item {} 
1       T       balsamic vinegar

\item {} 
1       t       salt

\item {} 
1       t       olive oil

\item {} 
1           large cucumber, diced

\item {} 
4           cloves garlic, minced

\item {} 
1/2     C       walnuts, chopped

\item {} 
1/2     C       fresh dilled, chopped

\end{itemize}


\section{Directions}
\label{tarator:directions}\begin{enumerate}
\item {} 
Mix the yoghurt and water thoroughly.

\item {} 
Mix in the rest of the ingredients.

\item {} 
Chill and serve.

\end{enumerate}


\section{Notes}
\label{tarator:notes}
Serves 6. This is a Bulgarian recipe from my mother.


\chapter{Nut Butter Muffins}
\label{nut_butter_muffins::doc}\label{nut_butter_muffins:nut-butter-muffins}

\section{Ingredients}
\label{nut_butter_muffins:ingredients}\begin{itemize}
\item {} 
1   C   nut butter such as peanut or almond

\item {} 
2       spotty bananas

\item {} 
2       large eggs

\item {} 
1/2 t   baking powder

\item {} 
1   t   apple cider vinegar

\item {} 
1   C   blueberries

\end{itemize}


\section{Directions}
\label{nut_butter_muffins:directions}\begin{enumerate}
\item {} 
Preheat the oven to 180 C.

\item {} 
Mix the nut butter, bananas, and eggs until smooth.

\item {} 
Mix the baking powder and vinegar in a separate bowl and then add that to the main mix.

\item {} 
Gently fold in the blueberries.

\item {} 
Spoon the batter into 12 muffin cup and bake for 10 minutes.

\end{enumerate}


\chapter{Chocolate Chip Cookies}
\label{chocolate_chip_cookies:chocolate-chip-cookies}\label{chocolate_chip_cookies::doc}

\section{Ingredients}
\label{chocolate_chip_cookies:ingredients}\begin{itemize}
\item {} 
4       Medjool dates, pits removed

\item {} 
2   C   almonds

\item {} 
1   t   baking soda

\item {} 
1/8     t   salt

\item {} 
1   t   vanilla extract

\item {} 
2   T   coconut oil, melted

\item {} 
1       egg

\item {} 
1/4     C   shredded coconut

\item {} 
1/2     C   dark chocolate chips

\end{itemize}


\section{Directions}
\label{chocolate_chip_cookies:directions}\begin{enumerate}
\item {} 
Preheat the oven to 175 Celcius.

\item {} 
In a food processor, grind the dates, almonds, baking soda, salt, and vanilla into a fine mix.

\item {} 
Add the coconut oil and egg, and blend for a few more seconds until mixed.

\item {} 
Transfer the mix to a bowl, and stir in the shredded coconut and chocolate chips.

\item {} 
Shape the mix into roughly 12 cookies and place them on a baking sheet.

\item {} 
Bake until golden brown, roughly 15 minutes.

\end{enumerate}


\section{Notes}
\label{chocolate_chip_cookies:notes}
Makes about 12 cookies.


\chapter{Chocolate Peanut Butter Banana Pie}
\label{cpbb_pie:chocolate-peanut-butter-banana-pie}\label{cpbb_pie::doc}

\section{Ingredients}
\label{cpbb_pie:ingredients}\begin{itemize}
\item {} 
1   C   diced almonds

\item {} 
1/4     C   coconut oil

\item {} 
1       banana, sliced into disks

\item {} 
250     g   dark chocolate

\item {} 
1/2     C   peanut butter

\item {} 
500     g   silken tofu

\item {} 
4       T   maple syrup or honey

\end{itemize}


\section{Directions}
\label{cpbb_pie:directions}\begin{enumerate}
\item {} 
Crust.  Melt the coconut oil and mix in the almonds.

\item {} 
Press into a pie plate.

\item {} 
Lay the banana slices on the crust.

\item {} 
Filling.  Melt the chocolate in a double-boiler.

\item {} 
Mix the melted chocolate, peanut butter, tofu, and maple syrup into a homogenous goo.

\item {} 
Pour the goo into the crust and chill for an hour.

\end{enumerate}


\chapter{Apricot Balls}
\label{apricot_balls:apricot-balls}\label{apricot_balls::doc}

\section{Ingredients}
\label{apricot_balls:ingredients}\begin{itemize}
\item {} 
3/4     C   raw cashews

\item {} 
3/4     C   raw almonds

\item {} 
1/4     C   pitted Medjol dates

\item {} 
3/2     C   dried apricots

\item {} 
1   dash salt

\item {} 
1/4     C   shredded coconut

\item {} 
1   T   grated orange zest

\item {} 
3/2     T   grated fresh ginger

\end{itemize}


\section{Directions}
\label{apricot_balls:directions}\begin{enumerate}
\item {} 
In a food processor, blend all the ingredients except the ginger until homogeneous.

\item {} 
Blend in the ginger.

\item {} 
Shape into balls.

\end{enumerate}


\section{Notes}
\label{apricot_balls:notes}
Makes about 20 tablespoon-sized balls.


\chapter{Ice Cream}
\label{ice_cream:ice-cream}\label{ice_cream::doc}

\section{Ingredients}
\label{ice_cream:ingredients}\begin{itemize}
\item {} 
4         eggs yolks

\item {} 
500   mL  cream

\item {} 
1/2   C   maple syrup or honey

\item {} 
1/4   t   guar gum (to prevent the ice cream from getting too hard)

\item {} 
flavorings

\end{itemize}


\section{Directions}
\label{ice_cream:directions}\begin{enumerate}
\item {} 
Mix all the Ingredients in a bowl.

\item {} 
Pour into an ice cream maker and make. Alternatively, put in the freezer and stir every 20 minutes until it reaches your desired consistency.

\end{enumerate}


\section{Notes}
\label{ice_cream:notes}
Here are some example flavorings.
\begin{itemize}
\item {} 
Vanilla: add 2 t vanilla extract

\item {} 
Chocolate: add 1 t vanilla extract and 150 g melted dark chocolate

\item {} 
Raspberry: add 2 t vanilla extract and 300g mashed raspberries

\item {} 
Peppermint: add 2 t peppermint extract and 75 g chocolate chips

\item {} 
Rosey: add 2 T rosewater

\item {} 
Fat Elvis: add 2 t vanilla extract, 1.5 very ripe bananas mashed, 75 g chocolate chips, and 1/2 C peanut butter

\end{itemize}


\chapter{Anise Drink}
\label{anise_drink:anise-drink}\label{anise_drink::doc}

\section{Ingredients}
\label{anise_drink:ingredients}\begin{itemize}
\item {} 
1   L   water

\item {} 
1   T   anise seeds

\item {} 
1   T   flax seeds

\item {} 
1   T   honey

\item {} 
1       lemon, juiced

\end{itemize}


\section{Directions}
\label{anise_drink:directions}\begin{enumerate}
\item {} 
Boil the water and anise seeds for 20 minutes.

\item {} 
Add the flax seeds and boil for another 10 minutes.

\item {} 
Remove from heat, strain out the seeds, and stir in honey and lemon.

\end{enumerate}


\chapter{In Praise of Baking Soda}
\label{baking_soda::doc}\label{baking_soda:in-praise-of-baking-soda}
Baking soda, also known as sodium bicarbonate, sodium hydrogen carbonate, and NaHCO3, is a versatile chemical compound.  Besides baking with it, you can use it as part of
\begin{itemize}
\item {} 
Tooth powder: mix three parts baking soda and one part salt and brush your - teeth with the stuff

\item {} 
Mouthwash: add baking soda to water and rinse to neutralize mouth acids and - kill bacteria

\item {} 
Body deodorant: dampen your armpits with water and apply a little baking soda

\item {} 
Exfoliant: dampen your skin with water and gently rub with baking soda

\item {} 
Cleaner: sprinkle some baking soda on a dirty surface, add your favorite - liquid cleaner (water, vinegar, etc.), and rub.  Don’t do this on aluminum surfaces, though, as baking soda attacks the thin nonreactive protective oxide layer of this otherwise very reactive metal.

\end{itemize}

For more uses of baking soda, check out its \href{http://en.wikipedia.org/wiki/Baking\_soda}{Wikipedia article}.



\renewcommand{\indexname}{Index}
\printindex
\end{document}
